%
% Kapitel Leistungsbewertung
%
%

\chapter{Leistungsbewertung}

\section{Laufzeit- und Strukturvergleich}
\subsection{TPC-D Q2}
Findet den g�nstigsten Lieferanten f�r ein gegebenes Bauteil in einer Region. F�r ein Bauteil einer bestimmten Art und Gr��e wird der Lieferant ermittelt, der dieses zu den g�nstigsten Konditionen anbietet. Gibt es mehrere Lieferanten in der gew�hlten Region, die das gew�nschte Teil zu den g�nstigsten Konditionen anbieten, so werden die 100 Lieferanten mit dem h�chsten Kontostand angezeigt. Von jedem Lieferanten werden der Kontostand, der Name und das Land, die Teilenummer und der Hersteller, die Lieferantenadresse, Telefonnummer und ein Kommentar angezeigt. Als Werte der Parameter SIZE, TYPE und REGION wurden die Werte aus der Abfrage�berpr�fung des Benchmarks gew�hlt.
\begin{itemize}
	\item SIZE = 15
	\item TYPE = BRASS
	\item REGION = EUROPE
\end{itemize}
\begin{lstlisting}[caption=Abfrage Q2]
select
	[sacctbal, sname,
	 nname, ppartkey,
	 pmfgr, saddress,
	 sphone, scomment]
from
	[part, supplier,
	 partsupp, nation, region]
where
	[ppartkey = pspartkey,
	ssuppkey = pssuppkey,
	psize = 15,
	ptype contains "BRASS",
	snationkey = nnationkey,
	nregionkey = rregionkey,	
  rname = "EUROPE",
	pssupplycost = (
		select
			min(ps:pssupplycost)
		from
			[partsupp as ps, supplier as s,
			nation as n, region as r]
		where
			[ppartkey = ps:pspartkey,
		 	s:ssuppkey = ps:pssuppkey,
		 	s:snationkey = n:nnationkey,
		 	n:nregionkey = r:rregionkey,	
		 	r:rname = "EUROPE"]
	 )]
orderby[sacctbal desc,
		 nname,
	   sname,
	   ppartkey] 
first 100
\end{lstlisting}

Die erste tempor�re Relation ist die Restriktion der im inneren Abfrageblock verwendeten Relationen des �u�eren Blocks auf die ben�tigten Attribute.

\begin{lstlisting}[caption=TempRel1]
select distinct[ppartkey, pspartkey] from [part, partsupp]
\end{lstlisting}

Einschr�nkung der inneren Relation durch alle \emph{simplen} Pr�dikaten.

\begin{lstlisting}[caption=TempRel2]
select*
from[partsupp as ps, 
		supplier as s, 
		nation as n, 
		region as r]
where[s:ssuppkey = ps:pssuppkey, 
		s:snationkey = n:nnationkey, 
		n:nregionkey = r:rregionkey, 
		r:rname = "EUROPE"]
\end{lstlisting}

Join der �u�eren und inneren Relationen und Berechnung der Aggregation. Da sich Optimierer und ausf�hrbare Ebene in der Syntax f�r umbenannte Tupel unterscheiden muss hier auf die ausf�hrbare Syntax der Spalten zur�ck gegriffen werden.

\begin{lstlisting}[caption=TempRel3]
select[pspartkey_ps, min(pssupplycost_ps) as var1]
from [txxrel1, txxrel2]
where [ppartkey = pspartkey_ps]
groupby pspartkey_ps
\end{lstlisting}

\begin{lstlisting}[caption=Entschachtelte Variante von Abfrage Q2]
select
	[sacctbal, sname, 
	nname, ppartkey, 
	pmfgr, saddress, 
	sphone, scomment]
from
	[part, supplier, 
	partsupp, nation, 
	region, txxrel3 as txxrel4]
where
	[ppartkey=pspartkey, 
	ssuppkey=pssuppkey, 
	psize=15, 
	ptype contains "BRASS", 
	snationkey=nnationkey, 
	nregionkey=rregionkey, 
	rname="EUROPE", 
	ppartkey=txxrel4:pspartkey_ps, 
	pssupplycost=txxrel4:var1]
orderby[sacctbal desc, 
		 nname, 
		 sname, 
		 ppartkey]
first 100
\end{lstlisting}

\subsection{TPC-D Q4}
Diese Abfrage ermittelt, wie gut das 
\begin{lstlisting}[caption=Abfrage Q4]
select
	[oorderpriority,
	 count(*) as ordercount]
from orders
where		
	[oorderdate >= instant("1993-07-01"),		
	oorderdate < theInstant(
		year_of(instant("1993-07-01")), 
	 	month_of(instant("1993-07-01")) + 3, 
	 	day_of(instant("1993-07-01"))
	),
	exists(
	 	select *
		from lineitem
		where
			[lorderkey = oorderkey, lcommitdate < lreceiptdate]
	)]
groupby [oorderpriority]
orderby	[oorderpriority]
\end{lstlisting}

Projektion der �u�eren Relation auf die Join-Spalte.
\begin{lstlisting}[caption=TempRel1]
select distinct[oorderkey] from orders
\end{lstlisting}

Einschr�nkung der inneren Relation mit allen \emph{simplen} Pr�dikate.
\begin{lstlisting}[caption=TempRel2]
select * from lineitem where[lcommitdate < lreceiptdate]
\end{lstlisting}


Full-Outerjoin zwischen den tempor�ren Relationen. Das \texttt{ifthenelse} Konstrukt ist notwendig, da \textsc{SECONDO} nur die zeilenbasierte Semantik des \texttt{count}-Operators unterst�tzt. 
\begin{lstlisting}[caption=TempRel3]
txxrel1  feed 
txxrel2  feed  
smouterjoin[oORDERKEY,lORDERKEY] 
extend[var1: ifthenelse(isempty(.lORDERKEY), 0, 1)] 
sortby[oORDERKEY asc] 
groupby[oORDERKEY;Var1: group feed sum[var1]] 
projectextend[Var1; lORDERKEY: .oORDERKEY] 
consume
\end{lstlisting}

\begin{lstlisting}[caption=Entschachtelte Variante von Q4]
select [oorderpriority, 
	count(*)as ordercount] 
from [orders, 
	txxrel3 as txxrel5] 
where [oorderdate>=instant("1993-07-01"), 
	oorderdate<theInstant(year_of(instant("1993-07-01")), 
	month_of(instant("1993-07-01")) + 3, 
	day_of(instant("1993-07-01"))), 
	oorderkey = txxrel5:lorderkey, 
	0 < txxrel5:var1] 
groupby [oorderpriority] 
orderby [oorderpriority]
\end{lstlisting}

\subsection{TPC-D Q7}
\subsection{TPC-D Q9}
\subsection{TPC-D Q15}
\subsection{TPC-D Q16}
\subsection{TPC-D Q17}
\subsection{TPC-D Q20}
\subsection{TPC-D Q21}
\subsection{TPC-D Q22}
\begin{table}[ht]
\centering
\begin{tabular}{@{}lrr@{}} \toprule
& \multicolumn{2}{c}{Laufzeit in s} \\ \cmidrule(l){2-3}
Abfrage Nr. & iterativ & entschachtelt\\ \midrule
Q2 & & \\
Q4 & & \\
Q7 & & \\
Q9 & & \\
Q15 & & \\
Q16 & & ---\\
Q17 & & \\
Q20 & & \\
Q21 & & \\
Q22 & & \\ 
\bottomrule\end{tabular}
  \caption{Laufzeit \enquote{kalt}}
  \label{tab_runtime_with_selectivity}
\end{table}

\begin{table}[ht]
\centering
\begin{tabular}{@{}crr@{}} \toprule
& \multicolumn{2}{c}{Laufzeit in s} \\ \cmidrule(l){2-3}
Abfrage Nr. & iterativ & entschachtelt\\ \midrule
Q2 & 5.347& 3.070\\
Q4 & 133.000& 0.643\\
Q7 & 2.759& 0.099\\
Q9 & & \\
Q15 & & \\
Q16 & & ---\\
Q17 & 0.588& 21.382\\
Q20 & 1.297& 0.506\\
Q21 & 1.381& 3.220\\
Q22 & 2.308& 0.938\\ 
\bottomrule\end{tabular} 
	\caption{Laufzeit \enquote{warm}}
  \label{tab_runtime}
\end{table}

\subsection{Fazit}
\subsection{Ausblick}
allgemeiner Full-Outerjoin-Operator auf Basis von Symmjoin. Unterst�tzung von \textbf{GROUPBY...HAVING}-Klauseln inkl. Subqueries. Geschachtelte Pr�dikate in Disjunktionen.

%
% EOF
%
%