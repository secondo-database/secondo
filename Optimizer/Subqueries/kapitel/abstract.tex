%
% abstract
%
%

\begin{abstract}
%  \chapter{Einleitung}
  \pagenumbering{arabic}
%    \section{Problembeschreibung}
%  \chapter{Review}
%    \section{�berblick}
%      \subsection{Beschreibung SECONDO}
%      \subsection{SQL}
%      \subsection{TPC-D Benchmark}
%    \section{Ausf�hrungsstrategien}
%      \subsection{Geschachtelte Iteration}
%      \subsection{Entschachtelung}
%     \section{Klassifikation geschachtelter Abfragen}
%     	\subsection{Typ A}
%     	\subsection{Typ N}
%     	\subsection{Typ J}
%     	\subsection{Typ JA}
%  \chapter{Entwurf}
%    \section{Optimierungsstrategie}
%    	\subsection{Einschr�nkungen}
%    \section{Algorithmus NEST-N-J}
%    \section{Algorithmus NEST-JA2}
%  \chapter{Implementierung}
%    \section{Ein Outerjoin Operator f�r SECONDO}
%    \section{Umschreiben quantifizierter Pr�dikate}
%    \section{Entschachtelung}
%    \section{Schema-Lookup}
%    \section{Ermittlung des Ausf�hrungsplans}
%    \section{�bersetzung des Plans in ausf�hrbare Syntax}
%  \chapter{Leistungsbewertung}
%    \section{Laufzeit- und Strukturvergleich}
%    	\subsection{TPC-D Q2}
%    	\subsection{TPC-D Q4}
%    	\subsection{TPC-D Q7}
%    	\subsection{TPC-D Q9}
%    	\subsection{TPC-D Q15}
%    	\subsection{TPC-D Q16}
%    	\subsection{TPC-D Q17}
%    	\subsection{TPC-D Q20}
%    	\subsection{TPC-D Q21}
%    	\subsection{TPC-D Q22}
%    \section{Testm�glichkeiten}
%    \section{Fazit}
%    \section{Ausblick}
\end{abstract}

%
% EOF
%
%