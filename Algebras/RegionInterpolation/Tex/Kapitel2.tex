
\chapter[Darstellung der Grundlagen \anmerkung{30-40 Seiten}]{Darstellung der Grundlagen\\
\normalsize{(nicht selbstgemachtes
Secondo,
MovingRegion,
T\o{}ssebro)}\anmerkung{30-40 Seiten}} \label{Kapitel2}
\minitoc
\newpage
\section{Secondo \anmerkung{10 Seiten}}
\anmerkung{Eine ausf"uhlichere Beschreibung von Secono, speziell des Konzeptes der Algebren, und eine Beschreibung der Algebren ,,Spatial-Algebra'' und ,Moving-Region-Algebra''.}
\subsection{Das Konzept der Secondo-Algebren}
\subsection{Spatial-Algebra}
\subsection{Moving-Region-Algebra}

\section{Das Paper von Erlend T\o{}ssebro \anmerkung{10 Seiten}}
\anmerkung{Hier beschreibe ich den Inhalt des Papers}

Eine der wichtigsten Grundlagen der vorliegenden Arbeit ist das Paper: ,,Creating Repesentations for Continuously Moving Regions from Observations'' von Erlend T\o{}ssebro und Ralf Hartmund G"uting \cite{TG}, in dem die Autoren sich mit den theoretischen Grundlagen des Problems besch"aftigt, und L"osungsvorschl"age zu vielen der vorkommenden Teilprobleme gemacht haben.

\subsection{Matching-Strategien}
In ihrem Paper nannten die Autoren drei verschiedene Matching-Strategien, die im folgenden kurz aufgef"uhrt werden:
\begin{enumerate}
\item Position of centroid

Bestimme den Schwerpunkt jedes Cycles,  bilde aus diesem einen gewichteten Graphen, mit den Entfernungen als Kantengewichte und suche in diesem "`N"achste Nachbarn"'.
\item Fixed threshold (set of cycles)

Matche zwei Cycles, wenn sie sich wechselseitig  mehr als threshold (in \%) "uberlappen.

\item Maximize Overlap (set of cycles)

Bilde einen gewichteten Graphen, in dem die Cycles Knoten sind, und dessen Kanten mit dem Grad der "Uberlappung gewichtet sind. Matche dann ein Cycle $c$ mit demjenigen, mit dem er die gr"o"ste "Uberlappung aufweist, und mit allen, f"ur die $c$ der Cycle mit der gr"o"sten "Uberlappung ist.
\end{enumerate} 

\section{Das Paper: ,,Matching Shapes with a Reference Point''  \anmerkung{10 Seiten}}

\anmerkung{Hier gebe ich eine Inhaltsangabe des Papers, und gehe speziell auf die Auswirkungen ein, die diese auf meine Matchingideen haben.}

In den Arbeiten \cite{AAR} und  \cite{AFRW} wurden der Schwerpunkt und der Steinerpunkt als m"ogliche Referenzpunkte gefunden. Ich fasse die Ergebnisse hier noch einmal zusammen:

\subsection{\index{Hausdorff-Abstand}Hausdorff-Abstand}

Seien $A$ und $B$ zwei kompakte Teilmengen des $\mathbb{R}^2$ und sei $\Vert\centerdot\Vert$ die Euklidische Norm.
Dann definieren wir eine Hilfsfunktion $ \widetilde{\delta_H}  $, den einseitige Hausdoff-Abstand, wie folgt:
\[ \widetilde{\delta_H}(A,B):=\max_{a\in A} \;\min_{b\in B} \Vert a-b \Vert\]
Der einseitige Hausdorff-Abstand von Polygon $A$ zu Polygon $B$ ist so definiert, dass er der Abstand des am weitesten entfernten Punktes aus $A$ zu dem im am n"achsten gelegenen Punkt aus $B$ ist.  Im weiteren Schritt wird der Hausdorff-Abstand wie folgt definiert: 
\[\delta_H:=\max\{\widetilde{\delta_H}(A,B),\widetilde{\delta_H}(B,A)\}.\]


\subsection{\index{Steiner-Punkt}Der Steiner-Punkt}

In \cite{Sch} wird der Steinerpunkt beschrieben "`als Schwerpunkt der Massenverteilung, die bei einem konvexen Polygon duch Belegung der Ecken mit den "au"seren Winkeln als Massen [...] gegeben ist"'. Folglich kann der Steiner-Punkt eines konvexen Polygones $P$, das aus den $n$ Eckpunkten $v_i$ besteht, berechnet werten ($\alpha_i$ ist hierbei der Innenwinkel von $v_i$):
\[p_2(P)=\sum^n_{i=1}v_i (\pi-\alpha_i).\]

\section{Das Paper: ,,Matching Convex Shapes with Respect to Symmetric Difference'' \anmerkung{10 Seiten}}
\anmerkung{Hier gebe ich eine Inhaltsangabe des Papers, und gehe speziell auf die Auswirkungen ein, die diese auf meine Matchingideen haben.}

\subsection{\index{symmetrische Differenz}Die symmetrische Differenz}

In \cite{AFRW} wird vorgeschlagen den Fl"acheninhalt der symmetrischen Differenz als Abstands-Ma"s zweier konvexer Polygone zu benutzen. Die Symmetrische Differenz von zwei kompakten Teilmengen $A$ und $B$ des $\mathbb{R}^2 $ ist definiert als:
\[A\bigtriangleup B:=(A\setminus B)\cup(B\setminus A).\]
Wenn $A(\cdot)$ der Fl"acheninhalt ist, so bildet $\delta_S$ den Abstand nach der symmetrischen Differenz:
\[\delta_S:=A(A \bigtriangleup B).\]

\subsection{\index{Schwerpunkt}Der Schwerpunkt}

Der Schwerpunkt eines Polygones ist der Schwerpunkt der Massenverteilung, die entsteht, wenn man allen Eckpunkten die selbe Masse zuordnet. Er berechnet sich f"ur ein Polygon $P$ mit $n$ Eckpunkten $v_i$:
\[p_0(P)=\sum^n_{i=1}v_i \frac{1}{n}.\]

%\printindex

