\section{Operations}
\label{sec:operations}
In this section, we introduce the operators defined on the proposed data types. 
We keep the name of the standard operators in \cite{GBE+00}, but the signatures and semantics 
are changed as the input data is equipped with novel knowledge. Section \ref{sec:extendedtypesystem}
introduces the type system in the general model. 
Operations on non-temporal and temporal generic data types are discussed in Sections \ref{sec:non-temporalop} and \ref{sec:temporaltypes}, respectively. Section \ref{sec:infrastructureandtm} considers operators on transportation modes and \textit{infrastructure objects}.  Section \ref{sec:datatypeconvert} presents operators for converting from generic data types to specific data types according to the infrastructure. In the end, Section \ref{sec:specific} deals with operators for some specific infrastructures.


\subsection{Type System}
\label{sec:extendedtypesystem}
\cite{GBE+00} provides a \textit{type system} for moving objects in free space shown in Table \ref{tab:typesystem1} and later \cite{GA2006} extends the system to support moving objects 
in road network shown in Table \ref{tab:typesystem2}. In general, the type system is specified as a signature which consists of \textit{sorts} and \textit{operations}. 
The \textit{sorts} are \textit{kinds} and represent sets of data types, and  
\textit{operations} are \textit{type constructors}. Type constructors may take arguments or not, in the latter case they are \textit{constant types}. The terms generated by the signature
describe the types available in the type system. More background refers to \cite{G1993}.

\begin{table}[ht]
 \begin{center} 
  \begin{tabular}{c|l|l}
	\hline
	\multirow{3}{*}{}&$\rightarrow BASE$&$\underline{int}$, $\underline{real}$, $\underline{string}$, $\underline{bool}$ \\
	& $\rightarrow SPATIAL$& $\underline{point}$, $\underline{points}$, $\underline{line}$, $\underline{region}$\\
	& $\rightarrow TIME$& $\underline{instant}$\\
	\hline
	$BASE\cup SPATIAL$&$\rightarrow TEMPORAL$&$\underline{moving}$, $\underline{intime}$\\
	\hline
	$BASE\cup TIME$&$\rightarrow RANGE$&$\underline{range}$\\
	\hline
  \end{tabular}
 \end{center}
 \caption{\label{tab:typesystem1}Type System in \cite{GBE+00}}
\end{table}


\begin{table}[ht]
 \begin{center} 
  \begin{tabular}{c|l|l}
	\hline
	\multirow{5}{*}{}&$\rightarrow BASE$&$\underline{int}$, $\underline{real}$, $\underline{string}$, $\underline{bool}$ \\
	& $\rightarrow SPATIAL$& $\underline{point}$, $\underline{points}$, $\underline{line}$, $\underline{region}$\\
	& $\rightarrow GRAPH$& $\underline{gpoint}$, $\underline{gline}$\\
	& $\rightarrow TIME$& $\underline{instant}$\\
	\hline
	$BASE\cup SPATIAL \cup GRAPH$&$\rightarrow TEMPORAL$&$\underline{moving}$, $\underline{intime}$\\
	\hline
	$BASE\cup TIME$&$\rightarrow RANGE$&$\underline{range}$\\
	\hline
  \end{tabular}
 \end{center}
 \caption{\label{tab:typesystem2}Type System in \cite{GA2006}}
\end{table}

We summarize the data types proposed in this paper listed in Table \ref{tab:summarydatatype} which are classified into three groups. The first group is data types applying for all infrastructures. The second and third groups are specific data types for two infrastructures:
$I_{ptn}$ and $I_{indoor}$. The data type $\underline{mpptn}$ for bus trips (introduced in Section \ref{sec:dynamic}) applies the framework of generic moving objects ($\underline{genmo}$) so that it does not exist in the table and the type system introduced later.\\

We give the type system for the general model in Table \ref{tab:typesystem3}. To make the system concise, the data type $\underline{tm}$ we define for transportation modes in Definition \ref{tm} can 
directly use $\underline{int}$ or $\underline{real}$, thus no new type has to be involved to represent transportation modes. Compared with \cite{GBE+00,GA2006}, there are three differences. 
\begin{itemize}
 \item To represent a location we can apply $\underline{genloc}$ in all cases while \cite{GBE+00} used $\underline{point}$ in free space, \cite{GA2006} introduced $\underline{gpoint}$ 
for road network. And $\underline{point3d}$ can be used for \textit{indoor}. Similarly, to represent a set of possible locations for moving objects we use $\underline{genrange}$ while \cite{GBE+00} used $\underline{line}$, \cite{GA2006} added $\underline{gline}$ and $\underline{line3d}$ is proposed for \textit{indoor}. With $\underline{genloc}$ and $\underline{genrange}$, we can cover the locations in all cases instead of defining different data types according to 
each specific environment. 

  \item Two groups of new data types \textit{PTN (Public Transportation Network)} and 
\textit{Spatial3D} are added for two specific infrastructures. The representation
for $\underline{busstop}$ uses the \textit{reference} method which is also utilized in \cite{GA2006} for identifying a location in road network. Due to the characteristics of \textit{PTN}, we introduce
$\underline{busroute}$ to describe all possible locations for dynamic 
infrastructure objects (moving buses). 
The \textit{indoor} environment processes 3D data. To distinguish with 
current 2D spatial data types, e.g., $\underline{point}$, $\underline{line}$, we call it \textit{Spatial3D}. For the data type $\underline{door}$ (Definition \ref{door}), it can be constructed by existing data types. We can use $\underline{genrange}$ to denote the two positions for a door and the types for the other two attributes (time-dependent state and the type of the door) are already in the type system. 

  \item For the moving types except those \textit{lifting} from
\textit{BASE}, e.g., $\underline{int}$, $\underline{real}$, $\underline{genloc}$ applies in all 
cases. That is the type $\underline{genmo}$ applies for moving objects in all infrastructures no matter whether the environment is free space, road network, or indoor. The case for dynamic infrastructure objects (i.e., buses, trains) is also included. This is because according to Section \ref{publictn}, the location of a bus is represented by its relative position in
a bus route which is consistent with $\underline{genloc}$.

\end{itemize}


\begin{table}[ht]
 \begin{center} 
  \begin{tabular}{c|c|c}
    \hline
    &\textbf{Name} & \textbf{Meaning}\\
    \hline
    \multirow{7}{*}{}generic data types &$\underline{tm}$& transportation modes\\
    &$\underline{genloc}$ & generic locations \\
    &$\underline{genrange}$ & generic trajectories \\
    &$\underline{genmo}$ & generic moving objects \\
    \hline 
    public transportation network&$\underline{busstop}$ & bus stops \\
    &$\underline{busroute}$ & bus routes \\
    \hline 
    indoor &$\underline{groom}$ & general rooms\\
    &$\underline{door}$ & doors\\
    &$\underline{point3d}$ & a position in indoor\\
    &$\underline{line3d}$ & indoor trajectories\\
    \hline 
  \end{tabular}
 \end{center}
 \caption{\label{tab:summarydatatype}A Summary of Proposed Data Types}
\end{table}


\begin{table}[ht]
 \begin{center} 
  \begin{tabular}{c|l|l}
	\hline
	\multirow{7}{*}{}&$\rightarrow BASE$&$\underline{int}$, $\underline{real}$, $\underline{string}$, $\underline{bool}$ \\
	& $\rightarrow SPATIAL$& $\underline{region}$\\
	& $\rightarrow TIME$& $\underline{instant}$\\
	& $\rightarrow SPACE$& $\underline{genloc}$, $\underline{genrange}$\\
    & $\rightarrow PTN$& $\underline{busstop}$, $\underline{busroute}$ \\
    & $\rightarrow Spatial3D$ & $\underline{groom},\underline{point3d},\underline{line3d}$ \\
	\hline
	$BASE\cup \{\underline{genloc}\}$&$\rightarrow TEMPORAL$&$\underline{moving}$, $\underline{intime}$\\
	\hline
	$BASE\cup TIME$&$\rightarrow RANGE$&$\underline{range}$\\
	\hline
  \end{tabular}
 \end{center}
 \caption{\label{tab:typesystem3}Type System in General Model}
\end{table}

\subsection{Operations on Non-Temporal Types}
\label{sec:non-temporalop}
The non-temporal operators applicable to generic data types  
are collected in Table \ref{tab:operators1}. Note that all these operators are still available 
for the earlier type systems. We believe the meaning for the predicates
should be obvious and give some brief comments. For predicates \textbf{intersects, inside},
the processing is more efficient if the result is \textit{false}, because with our method one 
can first check two objects' identifiers. If they are not equal, it can terminate and return
\textit{false}. Before, it may involve costly geometry computation. This thing also
applies for the group of set operations. As a $\underline{genrange}$
object is a set of elements, \textbf{no\_components} returns the number of elements (referenced infrastructure objects), e.g., how many streets it covers or how many rooms it passes. 
The operator \textbf{length} gets the geometry line length in space. 
To compute the distance and direction between $\underline{genloc}$ and $\underline{genrange}$ objects, 
the location has to be represented in the same coordinate system,
thus $\underline{space}$ is required as the input for transformation. Operator \textbf{ref\_id} returns the referenced object identifier. The type $\underline{genrange}$ represents a set of objects so that the result is a set of $\underline{int}$ values. 


\begin{table}[ht]
 \begin{center} 
 %\scalebox{0.7}{} 
  \begin{tabular}{c|c|l}
	\hline
    &Name & Signature \\
	\hline
	\multirow{2}{*}{Predicates, unary}&\textbf{isempty}&$\underline{tm} \rightarrow {\underline{bool}}$ \\
    & &$ \underline{genloc} \rightarrow {\underline{bool}}$ \\
	& &$ \underline{genrange} \rightarrow {\underline{bool}}$ \\
	\hline
	\multirow{4}{*}{Predicates, binary}&\textbf{$=,\neq$}&$ \underline{tm} \times \underline{tm} \rightarrow \underline{bool}$ \\
    & &$ \underline{genloc} \times \underline{genloc} \rightarrow \underline{bool}$ \\
	& &$ \underline{genrange} \times \underline{genrange} \rightarrow \underline{bool}$ \\
	& \textbf{intersects, inside}&$ \underline{genloc} \times \underline{genrange} \times \underline{space} \rightarrow \underline{bool}$ \\
	& &$ \underline{genrange} \times \underline{genrange} \times \underline{space} \rightarrow \underline{bool}$ \\
	\hline
	\multirow{4}{*}{Set operations}&\textbf{intersection}&$ \underline{genloc} \times 
	\underline{genrange} \times \underline{space} \rightarrow \underline{genloc}$ \\
	& &$ \underline{genrange} \times \underline{genrange} \times \underline{space} \rightarrow \underline{genrange}$ \\
	& \textbf{union}&$ \underline{genrange} \times \underline{genrange} \times \underline{space} \rightarrow \underline{genrange}$ \\
	& \textbf{minus}&$ \underline{genrange} \times \underline{genrange} \times \underline{space} \rightarrow \underline{genrange}$ \\
	\hline
	\multirow{2}{*}{Numeric}&\textbf{no\_components}&$ \underline{genrange} \rightarrow {\underline{int}}$ \\
	&\textbf{length} &$ \underline{genrange} \rightarrow {\underline{real}}$ \\
	\hline
	\multirow{2}{*}{Distance and Direction}&\textbf{distance}&$ \underline{genloc} \times  \underline{genloc} \times \underline{space} \rightarrow {\underline{real}}$ \\
	&&$ \underline{genloc} \times  \underline{genrange} \times \underline{space} \rightarrow {\underline{real}}$ \\
	&&$ \underline{genrange} \times  \underline{genrange} \times \underline{space} \rightarrow {\underline{real}}$ \\
	&\textbf{direction}&$ \underline{genloc} \times  \underline{genloc} \times \underline{space} \rightarrow {\underline{real}}$ \\
	\hline
    \multirow{2}{*}{Reference Id}&\textbf{ref\_id}&$ \underline{genloc} \rightarrow \underline{int}$ \\
    &&$ \underline{genrange} \rightarrow set(\underline{int})$ \\
    \hline
  \end{tabular}
 \end{center}
 \caption{\label{tab:operators1}Non-Temporal Generic Operators }
\end{table}

\subsection{Operations on Temporal Types}
\label{sec:temporaltypes}
Table \ref{tab:operators2} lists operators on temporal generic data types. 
Because we represent the location data in a multiresolution way where both imprecise and 
precise data is managed, all the operators defined in \cite{GBE+00,GA2006} 
for moving points can be applied, e.g., \textbf{passes}, \textbf{present}, but with 
the modification that in the signature $\underline{mpoint}$ or 
$\underline{mgpoint}$ is replaced by $\underline{genmo}$. \textbf{Trajectory} projects
a moving object into the infrastructure space and \textbf{deftime} yields the time intervals when 
the object is defined. The type $\underline{periods}$ is introduced in Section \ref{sec:preliminary} which is a set of time intervals. \textbf{duration} returns the time span of a period and we assume the result unit is \textit{minute}. For example, given a period (``2010-12-5-8-30'', ``2010-12-5-10-30''), \textbf{duration} returns 120. Section \ref{sec:multi-scaleresolution} introduces the method of representing moving objects' location in an imprecise way. We can get the moving object with coarse location representation by \textbf{lowres}. And then we can apply operator \textbf{trajectory} to get the rough trajectory. \\

Operator \textbf{passes} checks whether the moving object
passes a place or not. The place could be a set of locations or one location. For example, ``Did \textit{Bobby} pass \textit{Alexander} street?'' or ``Did \textit{Bobby} visit the Sparkasse Bank in city center?''. The place needs to be checked could also be a region in free space or region-based outdoor. \textbf{At} restricts the object within the given space. The space can be a point in space, then the movement is only in temporal, e.g., $t_{move}$ for buses introduced in Section \ref{sec:dynamic}. One can also obtain the sub movement inside a region where the object identifier in $\underline{genloc}$ maps to a region. 
\textbf{Present} checks whether the object is defined by the given time parameter. 
\textbf{Initial} and \textbf{final} yields the first and last value. The object can be restricted by
a time instant or a set of time intervals by \textbf{atinstant} and \textbf{atperiods}. \textbf{val} returns $\underline{genloc}$ from an $\underline{instant}$($\underline{genloc}$) object and \textbf{inst} returns the time. \\


Operator \textbf{decompose} transforms a generic moving objects into a set of values each of which has  one unique reference id. That means the location of each value (a moving object) in the result set  only maps to one infrastructure object. For example, if a pedestrian walks through a set of regions, \textbf{decompose} returns a set of sub movements where each corresponds to the movement inside one region.  The meaning of \textbf{speed} is clear and \textbf{mdirection} returns the time-dependent direction of movement.  

\begin{table}[ht]
 \begin{center} 
 %\scalebox{0.7}{} 
  \begin{tabular}{c|c|l}
	\hline
    &Name & Signature \\
	\hline
	\multirow{4}{*}{Projection to domain and range}&\textbf{trajectory}&$\underline{genmo} \rightarrow {\underline{genrange}}$ \\
	&\textbf{deftime} &$ \underline{genmo} \rightarrow \underline{periods}$ \\
    &\textbf{duration} &$ \underline{periods} \rightarrow \underline{real}$ \\
    &\textbf{lowres} &$\underline{genmo} \rightarrow \underline{genmo}$ \\
	\hline
	\multirow{5}{*}{Interaction to domain and range}&\textbf{passes}&
	$\underline{genmo} \times \underline{genloc} \times \underline{space} \rightarrow {\underline{bool}}$\\
	& &$\underline{genmo} \times \underline{genrange} \times \underline{space} \rightarrow {\underline{bool}}$\\
	&\textbf{at} &$\underline{genmo} \times \underline{genloc} \rightarrow {\underline{genmo}}$\\
    & &$\underline{genmo} \times \underline{genrange} \rightarrow {\underline{genmo}}$\\
	&\textbf{present}&$\underline{genmo} \times \underline{instant} \rightarrow {\underline{bool}}$ \\
    & &$\underline{genmo} \times \underline{periods} \rightarrow {\underline{bool}}$ \\
	&\textbf{initial, final} &$\underline{genmo} \rightarrow {\underline{intime}}({\underline{genloc}})$ \\
	&\textbf{atinstant} &$\underline{genmo} \times \underline{instant} \rightarrow {\underline{intime}}({\underline{genloc}})$ \\
	&\textbf{atperiods} &$\underline{genmo} \times \underline{periods} \rightarrow {\underline{genmo}}$ \\
    &\textbf{val} &${\underline{intime}}({\underline{genloc}}) \rightarrow \underline{genloc}$ \\
    &\textbf{inst} &${\underline{intime}}({\underline{genloc}}) \rightarrow \underline{instant}$ \\
    &\textbf{decompose} &$\underline{genmo} \rightarrow set(\underline{genmo}$) \\                
	\hline
	\multirow{1}{*}{Rate of change}&\textbf{speed, mdirection} &$\underline{genmo} \rightarrow {\underline{moving}(\underline{real})}$\\
	\hline
  \end{tabular}
 \end{center}
 \caption{\label{tab:operators2}Temporal Generic Operators }
\end{table}

\subsection{Transportation Modes and Infrastructure Objects}
\label{sec:infrastructureandtm}
Table \ref{tab:operators4} lists the proposed operators. Given a moving object, we can get its transportation modes which is a set of objects. According to different application requirement, the transportation modes 
can be sorted by a specified order. For example, modes can be increasingly ordered by the average speed value. Alternatively, we can also order them by appearing time in the moving object. Assuming we choose the time, use \textit{Bobby}'s trajectory $M_1$ as an example, \textbf{get\_mode} returns $<Walk, Car, Indoor>$. Given a transportation mode, the moving object can restricted to a sub movement according to that mode which is done by \textbf{at} (in Section \ref{sec:temporaltypes}, \textbf{at} rectricts the object to a given space and now it is extended). \textbf{contain} checks whether a specified mode is included by a set of transportation modes. With \textbf{get\_mode} and this operator, we can examine whether the moving object includes a specific transportation mode. For example, one can issue ``Does \textit{Bobby} use public transportation vehicles during this trip?''. \textbf{no\_of\_components} returns the number of transportation modes.\\


Before introducing the operators for referenced \textit{infrastructure objects}, we first define an reference data type for infrastructure objects named $\underline{ioref}$.  

\begin{Statement}
\label{infrastructuretype}
\ Reference Data Type

$D_{\underline{ioref}} = \{(oid,ref)|oid \in D_{\underline{int}}, ref \in IOSymbol\}$

\end{Statement}

The values of type $\underline{ioref}$ have two components where $oid$ is the referenced object id and $ref$ is the symbol of a data type (Definition \ref{infraobjectsymbol}). The data type is a compact representation for \textit{infrastructure objects} which have different data types according to the infrastructure and the value of which may need large storage space, e.g., region. With the reference type, we can describe the result in a light way. Operator \textbf{ref\_id} returns the referenced object id. When the underlying information is needed, one can get the full data by operator 
\textbf{ref\_obj}. One possibility could be \\

$(oid$, \textit{GROOM}) $\times \underline{space} \rightarrow \underline{groom}$. \\

Operator \textbf{get\_infra} gets the \textit{referenced infrastructure objects}. A simple case is only one infrastructure object is returned, e.g., a street or an office room. If the result is a set of \textit{infrastructure objects}, we can order them by the referenced time in the moving object. Use \textit{Bobby}'s trajectory $M_1$ as an example and \textbf{get\_infra} gets $<set((oid$, \textit{REGION})), $set((oid$, \textit{LINE})), $set((oid$, \textit{GROOM})) $>$ where the result is described by the reference data type. Applying operator \textbf{ref\_type}, 
we can also get the full representation: \\

$<set(IO(oid$, \textit{REGION}, $\beta$, $name)(\beta \in D_{\underline{region}}))$,  

\hspace{0.25cm} $set(IO(oid$, \textit{LINE}, $\beta$, $name)(\beta \in D_{\underline{line}}))$, 

\hspace{0.25cm} $set(IO(oid_3$, \textit{GROOM}, $\beta$, $name)(\beta \in D_{\underline{groom}}))>$. \\

Operator \textbf{contain} checks whether an object identifier is contained by a set of reference type objects. \textbf{no\_of\_components} returns the number of referenced infrastructure objects. One can combine this operator with \textbf{at} and \textbf{get\_infra} to get the number of referenced objects according to a specific transportation mode. For example, if the mode is $Bus$, \textbf{at} returns the moving object with the mode $Bus$. Then \textbf{get\_infra} takes the result and outputs the referenced buses. In the end, \textbf{no\_of\_components} returns the number of referenced objects which shows the times of bus transfer. \\

Finally, we define the operator \textbf{trip} for route planning as follows. \\

\textbf{trip}: $\underline{genloc} \times \underline{genloc} \times \underline{instant} \times \underline{space} \rightarrow \underline{genmo}$  \\

It takes two locations, a query instant time and the space as input where the location is general which can be in any \textit{infrastructure} proposed in this paper. The result is described in the form of $\underline{genmo}$. Paper \cite{BSWC09} gives the solution for
outdoor space with transportation modes constraint, while our graph model in Section \ref{sec:navigation} for $indoor$ navigation
can be considered as complementary so that it covers all environments. 
For example, ``find a trip from my office room to my home". The infrastructures covered by the result trip could be \textit{indoor}, \textit{road network}, \textit{region-based outdoor} 
($Indoor\rightarrow Car\rightarrow Walk$) or \textit{indoor}, \textit{region-based outdoor}, \textit{public transportation network} ($Indoor\rightarrow Walk\rightarrow Bus\rightarrow Walk$). 


\begin{table}[ht]
 \begin{center} 
 %\scalebox{0.7}{} 
  \begin{tabular}{c|c|l}
    \hline
    &Name & Signature \\
    \hline
    \multirow{4}{*}{Transportation Modes}& \textbf{get\_mode}&$ \underline{genmo} \rightarrow set(\underline{tm})$\\
    &\textbf{at}& $\underline{genmo}\times \underline{tm} \rightarrow \underline{genmo}$\\
    &\textbf{contain}& $set(\underline{tm}) \times \underline{tm} \rightarrow \underline{bool}$\\
    &\textbf{no\_of\_components}& $set(\underline{tm}) \rightarrow \underline{int}$\\
    \hline
    \multirow{3}{*}{Infrastructure Objects}& \textbf{ref\_id} & $\underline{ioref} \rightarrow \underline{int}$ \\ 
    &\textbf{ref\_obj} & $\underline{ioref} \times \underline{space} \rightarrow \alpha$ $(\alpha \in IOType)$ \\ 
    &\textbf{get\_infra}& $\underline{genloc} \times \underline{space} \rightarrow \underline{ioref}$ \\
    & &$\underline{genrange} \times \underline{space} \rightarrow set(\underline{ioref})$\\
    & &$\underline{genmo} \times \underline{space} \rightarrow set(\underline{ioref})$\\
    & \textbf{contain}& 
    $set(\underline{ioref}) \times \underline{int} \rightarrow \underline{bool}$\\
    & \textbf{no\_of\_components}& 
    $set(\underline{ioref}) \rightarrow \underline{int}$\\
    \hline
    trip planning& \textbf{trip}& $\underline{genloc} \times \underline{genloc} \times \underline{instant} \rightarrow \underline{genmo}$\\
    \hline 
  \end{tabular}
 \end{center}
 \caption{\label{tab:operators4}Operators on Transportation Modes and Infrastructure Objects}
\end{table}

\subsection{Data Type Converting Operations}
\label{sec:datatypeconvert}
In this subsection, we introduce operators converting from generic data types to data types for specific infrastructures (transportation modes). These data types are used to represent locations and trajectories of moving objects. Table \ref{tab:specificdatatyps} summarizes the data types and the corresponding transportation modes as well as the infrastructures. 
We can convert from generic data types to specifc data types according to the transportation mode. 
For the modes in \textit{public transportation network} and \textit{road network}, we use the same types (Note that $\underline{busstop}$ and $\underline{busroute}$ are used to represent the infrastructure objects and here $\underline{gpoint}$ and $\underline{gline}$ are used to represent the locations and trajectories of moving objects in the infrastructure). The representation for \textit{road network} should be clear. In \textit{public transportation network}, given a $gp_i(rid,pos)\in D_{\underline{gpoint}}$, $gp_i.rid$ denotes the bus route id and $gp_i.pos$ denotes the relative position on that route. In \textit{road network}, the type $\underline{gline}$ represents a set of route intervals $\{(rid_1,pos_1,pos_2)$, $(rid_2,pos_1,pos_2)$...$(rid_n,pos_1,pos_2)\}$ which can be used to represent a set of sub bus routes in \textit{public transportation network}. To let the convert operator general, we first define two sets named $Infraloc$ and $Infratraj$ each of which contains the specific data types as elements. . 

\begin{Statement}
\label{infrastructureloctraj}
\ Sets of Specific Data Types 

$Infraloc = \{\underline{point3d},\underline{point},\underline{gpoin}t\}$

$Infratraj = \{\underline{line3d},\underline{line},\underline{gline}\}$
\end{Statement}

Let $f_{loc}$ and $f_{traj}$ be two functions mapping from transportation modes to specific data types. \\

\begin{Statement}
\label{convertfunction}
\ Mapping from Transportation Modes to Specific Data Types 

$f_{loc}: D_{\underline{tm}} \rightarrow Infraloc$ 

$f_{traj}: D_{\underline{tm}} \rightarrow Infratraj$
\end{Statement}


Two examples are: (1) $f_{loc}(Indoor)=\underline{point3d}$; (2) $f_{traj}(Car)= \underline{gline}$. Two converting operators are listed in Table \ref{tab:convertoperators}.  

\begin{table}[ht]
 \begin{center} 
  \begin{tabular}{c|c|c}
    \hline
    \textbf{Infrastructures}& \textbf{Transportation Modes} & \textbf{Data Types} \\
    \hline
    Public Transportation Network&$Bus$, $Train$, $Tube$& $\underline{gpoint}$, $\underline{gline}$ \\
    \hline
    Indoor &$Indoor$ & $\underline{point3d}$, $\underline{line3d}$ \\
    \hline
    Region-based Outdoor&$Walk$ & $\underline{point}$ ,$\underline{line}$ \\
    \hline
    Road Network &$Car$, $Taxi$, $Bicycle$ & $\underline{gpoint}$, $\underline{gline}$ \\
    \hline
    Free Space&any mode & $\underline{point}$, $\underline{line}$\\
    \hline
  \end{tabular}
 \end{center}
 \caption{\label{tab:specificdatatyps}Specific Data Types for Locations and Trajectories}
\end{table}


\begin{table}[ht]
 \begin{center} 
  \begin{tabular}{c|c}
    \hline
    \textbf{Name} & \textbf{Signature} \\
    \hline 
    \textbf{convert\_loc}& For $\alpha \in D_{\underline{tm}}$, $\underline{genloc} \times \underline{space} \times \alpha \rightarrow f_{loc}(\alpha)$\\
    \hline 
    \textbf{convert\_traj}& For $\alpha \in D_{\underline{tm}}$, $\underline{genrange} \times \underline{space} \times \alpha \rightarrow f_{traj}(\alpha)$\\
    \hline 
  \end{tabular}
 \end{center}
 \caption{\label{tab:convertoperators} Converting Operators}
\end{table}


\subsection{Infrastructure Specific Operations}
\label{sec:specific}
In the end, we introduce operators for two specific infrastructures: public transportation network and indoor, listed in Table \ref{tab:operators3}. Operations for the infrastructure \textit{free space} and \textit{road network} are already in \cite{GBE+00,GA2006} and there are no new data types for the infrastructure \textit{Region-based Outdoor}. \textbf{geo\_data} returns the spatial attribute of a bus stop and a bus route and \textbf{no\_of\_components} returns the number of bus stops for a bus route. 
For \textit{indoor}, as $\underline{groom}$ represents a set of objects, \textbf{get\_region} and \textbf{get\_height} return a set of values. For a groom, we can get the quantity of its 3D regions and \textbf{length} returns the length of a 3D curve.

\begin{table}[ht]
 \begin{center} 
 %\scalebox{0.7}{} 
  \begin{tabular}{c|c|l}
    \hline
    &Name & Signature \\
    \hline
    \multirow{3}{*}{Public Transportation Network}&\textbf{geo\_data}&$ \underline{busstop} \times \underline{busroute} \rightarrow \underline{point}$ \\
    & &$ \underline{busroute} \rightarrow \underline{line} $ \\
    &\textbf{no\_components}&$ \underline{busroute} \rightarrow {\underline{int}}$ \\
    \hline
    \multirow{5}{*}{Indoor}&\textbf{get\_region}&$ \underline{groom} \rightarrow set(\underline{region})$ \\
    & \textbf{get\_height} &$ \underline{groom} \rightarrow set(\underline{real})$ \\
    &\textbf{no\_components}&$ \underline{groom} \rightarrow {\underline{int}}$ \\
    &\textbf{length}&$ \underline{line3d} \rightarrow {\underline{real}}$ \\
    \hline
  \end{tabular}
 \end{center}
 \caption{\label{tab:operators3}Operators for Specific Infrastructures}
\end{table}