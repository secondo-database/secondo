\section{Conclusions and Future Work}
\label{sec:conclusions}
In this paper, we propose a generic data model to manage moving objects in multiple
environments including public transportation network, \textit{indoor}, region-based outdoor, 
road network and free space. With this model the database system can manage the complete
movement of objects in different environments. We consider the \textit{space} where an object moves as a collection of so-called \textit{infrastructures}. 
Each \textit{infrastructure} corresponds to one moving environment and  
consists of a set of elements, called \textit{infrastructure objects}. Using the \textit{object-based} approach, we represent the location of moving objects by \textit{referencing} to these \textit{infrastructure} \textit{objects}. A generic model is presented where the location is 
expressed by a function from time to space which can be applied for all environments. 
Using the method of \textit{sliced representation}, we define a framework for generic 
moving objects. It is expressed in a multiresolution way where both approximate and accurate 
locations are supported.  In addition, semantic data (transportation modes) is encapsulated for moving objects making the data expressive which 
can be used for knowledge analysis. For each kind of \textit{infrastructure}, we define the corresponding data types representing the space it covers and the relative location inside, 
as well as the location expression for objects moving in that \textit{infrastructure}. A graph model is proposed for \textit{indoor} navigation where three types of optimal routes are supported: shortest distance, smallest number of rooms and minimum traveling time. The generic model encapsulates 
the existing work for free space and road network, and they are consistent. A set of operators are defined in the type system for generic model and a relational interface is provided for exchanging data between values of proposed data types and a relational environment. Finally, a group of example queries are formulated on the model. \\

We envision several future directions. The first is to implement the data model in an 
extensible database system Secondo including data types, operators, and visualization technology 
(3D data for \textit{indoor}). Another topic is to conduct trajectory pattern mining and analysis as
semantic data is involved. It is also interesting to consider a benchmark for generic moving objects. 

\section*{Acknowledgment}
\label{sec:acknowledgement}
The first author gratefully acknowledges the financial support by Chinese Scholarship Committee. 
