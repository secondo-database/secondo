\section{Query Examples}
\label{sec:exampledataandquery}
In this section, we evaluate our data model by performing a group of queries. We show 
how to use SQL-like language to formulate the queries 
expressed by natural language. We retain the same base syntax 
and structure of SQL language (most readers are familiar) with some extensions. 
The operators presented in Section \ref{sec:operations} can be registered as functions in the DBMS 
such that they are available for users. The earlier operators in \cite{GBE+00,GA2006} are all available.  Some of the queries are already supported in one specific
environment, but now we can express all of them in one data model instead of issuing the queries only in their environment. Before introducing the queries, we first explain some notations. \\

\texttt{LET} $<$ \texttt{name} $>$ = $<$ \texttt{query expression} $>$ or $<$ \texttt{object constructor} $>$\\

\texttt{ELEM} $<$ \texttt{rel} $>$ \\

\texttt{AGGR}($<$ \texttt{operator} $>$, $<$ \texttt{element} $>$) \\

The command \texttt{LET} creates a new object where the value comes from the query expression or an atomic type constructor. \texttt{ELEM} is used to covert a relation with a single tuple having the single attribute to an atomic value. For example, given a relation $r(Name:string)$ with only one tuple (``\texttt{Bobby}''), \texttt{ELEM$<$r$>$} gets ``\texttt{Bobby}''. 
The notation AGGR denotes an aggregate function on the value. 
Using the example data in Section \ref{sec:spacerel}, 
the queries we consider are listed and formulated as follows. We classify them into three groups (the classification may be not so strict because some queries cover more then one aspect) and there are more query examples in the Appendix. \\

\begin{itemize}
 \item Queries on Intrastructures and Infrastructure Objects
 \item Queries on Transportation Modes
 \item Interaction between Different Transportation Modes and Infrastructures \\
\end{itemize}



\textbf{1. Queries on Intrastructures and Infrastructure Objects} 

\begin{itemize}
 \item \textbf{Query} 1. Where is \textit{Bobby} at 8:00 am? \\

$\hspace{1cm}$ \texttt{LET qt = instant(2010, 12, 5, 8)}. \\


$\hspace{1cm}$ \texttt{SELECT \textbf{ref\_obj}(\textbf{get\_infra}(\textbf{val}(\textbf{atinstant}(mo.Traj, qt)), SpaceHagen))}

$\hspace{1cm}$ \texttt{FROM MOHagen as mo}

$\hspace{1cm}$ \texttt{WHERE mo.Name = ``Bobby''}
\end{itemize}

This query needs to combine four operators and we give some comments. First, after filtering the moving object by name we restrict the movement within the time instant done by operator \textbf{atinstant}. Second, we retrieve the value of an $\underline{intime}(\underline{genloc})$ object by \textbf{val}. Third, the \textit{infrastructure object} is returned by \textbf{get\_infra} and \textbf{ref\_obj} gets the full representation.  \\ 

\begin{itemize}
 \item \textbf{Query} 2. Find all people taking ``Bus527''. \\

\texttt{Let MBus527 =}

$\hspace{1cm}$ \texttt{SELECT *}

$\hspace{1cm}$ \texttt{FROM \textbf{get\_ptn}(SpaceHagen) as mbus}

$\hspace{1cm}$ \texttt{WHERE mbus.Name = ``Bus527''} \\


$\hspace{1cm}$ \texttt{SELECT mo.Name}

$\hspace{1cm}$ \texttt{FROM MOHagen as mo, MBus527 as mbus}

$\hspace{1cm}$ \texttt{WHERE \textbf{contain}(\textbf{get\_infra}(\textbf{at}(mo.Traj, Bus), SpaceHagen), mbus.Mp\_id)}


\end{itemize}

In the first step, we get the relation of all buses from the specified route. In the second step, we first restrict the movement with the mode $Bus$ and get the referenced objects (represented in a light way), and then we check whether the bus is referenced. \\

\begin{itemize}
 \item \textbf{Query} 3. Find who passes the room $r_{312}$ in FernUni (Fern University in Hagen) between 8:00 am and 9:00 am. \\

\texttt{LET qt = period(instant(2010, 12, 5, 8), instant(2010, 12, 5, 9))}. \\

Assume the name of the room is ``FernUni-312'' and first we get the room id by  \\

\texttt{LET room\_id = ELEM(}

$\hspace{1cm}$ \texttt{SELECT ind.In\_id}

$\hspace{1cm}$ \texttt{FROM \textbf{get\_indoor}(SpaceHagen) as ind}

$\hspace{1cm}$ \texttt{WHERE ind.Name = ``FernUni-312''}) \\


$\hspace{1cm}$ \texttt{SELECT mo.Name}

$\hspace{1cm}$ \texttt{FROM MOHagen as mo}

$\hspace{1cm}$ \texttt{FROM \textbf{contain}(\textbf{get\_infra}(\textbf{at}(\textbf{atperiods}(mo.Traj, qt), Indoor)}, 

$\hspace{4.6cm}$ \texttt{SpaceHagen), room\_id)}

\end{itemize}

To process this query, several operators are needed. First, we restrict moving objects within the given period by \textbf{atperiods}. Second, we get the referenced infrastructure objects which are only for \textit{indoor} by \textbf{at} and \textbf{get\_infra}. The result is represented by the reference type. Third, we check whether the room id is included by \textbf{contain}. \\

\begin{itemize}
 \item \textbf{Query} 4. Find all people passing zone $A$ and zone $B$ as well as a location $p$ 
in space. (An interesting query could be: find all people passing the Christmas Market Area and the city center area as well as the cinema.) \\

Assume the names for the zones are ``Zone-A'' and ``Zone-B''. Let $p(\bot, (x,y))(\in D_{\underline{genloc}})$ denote the location. First, we get the object ids for zones $A$ and $B$. \\ 

\texttt{LET A\_Id = ELEM(}

$\hspace{1cm}$ \texttt{SELECT rbo.Poly\_id}

$\hspace{1cm}$ \texttt{FROM \textbf{get\_rbo}(SpaceHagen) as rbo}

$\hspace{1cm}$ \texttt{FROM rbo.Name = ``Zone-A''}) \\

\texttt{LET B\_Id = ELEM(} 

$\hspace{1cm}$ \texttt{SELECT rbo.Poly\_id}

$\hspace{1cm}$ \texttt{FROM \textbf{get\_rbo}(SpaceHagen) as rbo}

$\hspace{1cm}$ \texttt{FROM rbo.Name = ``Zone-B''}) \\

Second, we create two $\underline{genloc}$ objects for the zones. \\

\texttt{LET IO\_A = genloc(A\_Id, ($\bot$, $\bot$))} \\

\texttt{LET IO\_A = genloc(A\_Id, ($\bot$, $\bot$))} \\

Third, we formulate the final query. \\ 

$\hspace{1cm}$ \texttt{SELECT mo.Name}

$\hspace{1cm}$ \texttt{FROM MOHagen as mo}

$\hspace{1cm}$ \texttt{FROM \textbf{passes}(mo.Traj, IO\_A, SpaceHagen) $\wedge$}

$\hspace{2cm}$ \texttt{\textbf{passes}(mo.Traj, IO\_B, SpaceHagen)  $\wedge$}

$\hspace{2cm}$ \texttt{\textbf{passes}(mo.Traj, p, SpaceHagen)} \\

\end{itemize}


\begin{itemize}
 \item \textbf{Query} 5. How long does \textit{Bobby} wait for bus 512 at the bus stop ``FernUniversity''?  \\

\texttt{LET bs\_stop = ELEM(} 

$\hspace{1cm}$ \texttt{SELECT bs.Stop}

$\hspace{1cm}$ \texttt{FROM \textbf{get\_bs}(SpaceHagen) as bs, \textbf{get\_br}(SpaceHagen) as br}

$\hspace{1cm}$ \texttt{WHERE bs.Name = ``FernUniversity'' $\wedge$ 
br.Name = ``Bus512'' $\wedge$}

$\hspace{2.1cm}$  \texttt{bs.Bs\_id = br.Br\_id})  \\

\texttt{LET bs\_route = ELEM(} 

$\hspace{1cm}$ \texttt{SELECT br.Route}

$\hspace{1cm}$ \texttt{FROM \textbf{get\_br}(SpaceHagen) as br}

$\hspace{1cm}$ \texttt{WHERE br.Name = ``Bus512''}) \\

We assume the travellers walk to the bus stop because normllay people do not switch from $Car$, $Taxi$, $Bicycle$ to $Bus$ according to the study in \cite{ZCXM09}. With the function \textbf{geo\_data} we can get the 2D point of the bus stop. \\

\texttt{LET p = geo\_data(bs\_stop, bs\_route)} \\


Then we create a $\underline{genloc}$ object for the bus stop denoted by $bs\_loc(\bot,(p.x,p.y))$. Then, the query is formulated by \\


$\hspace{1cm}$ \texttt{SELECT \textbf{duration}(\textbf{deftime}(\textbf{at}(\textbf{at}(mo.Traj, Walk), bs\_loc))}

$\hspace{1cm}$ \texttt{FROM MOHagen as mo}

$\hspace{1cm}$ \texttt{FROM mo.Name = ``Bobby''} \\

\end{itemize}

First, we get the sub movement for the mode is $Walk$. And then, the movement is retricted at the bus stop. Finally, \textbf{deftime} gets the time period at the place and \textbf{duration} returns the time span. \\


\textbf{2. Queries on Transportation Modes} \\

\begin{itemize}
 \item \textbf{Query} 6. At 8:00 am, who sits in the same bus as \textit{Bobby}. \\

\texttt{LET qt = instant(2010, 12, 5, 8)}\\

\texttt{LET bus\_id = ELEM(} 

$\hspace{1cm}$ \texttt{SELECT \textbf{ref\_id}(\textbf{val}(\textbf{atinstant}(mo.Traj, qt)))}

$\hspace{1cm}$ \texttt{FROM MOHagen as mo}

$\hspace{1cm}$ \texttt{FROM mo.Name = ``Bobby''}) \\


$\hspace{1cm}$ \texttt{SELECT mo.Name}

$\hspace{1cm}$ \texttt{FROM MOHagen as mo}

$\hspace{1cm}$ \texttt{FROM \textbf{ref\_id}(\textbf{val}(\textbf{atinstant}(mo, qt))) = bus\_id} \\
\end{itemize}

The FROM clause of the second query consists of three steps: 

\begin{itemize}
 \item restrict moving objects within the given time instant done by \textbf{atinstant};
 \item get the value from the $\underline{intime}(\underline{genloc})$ object, which is a $\underline{genloc}$ done by \textbf{val};
 \item get and compare the object identifier.  \\
\end{itemize}



\begin{itemize}
 \item \textbf{Query} 7. Find all people using public transportation vehicles. \\

$\hspace{1cm}$ \texttt{SELECT mo.Name}

$\hspace{1cm}$ \texttt{FROM MOHagen as mo}

$\hspace{1cm}$ \texttt{FROM \textbf{contain}(\textbf{get\_mode}(mo.Traj), Bus)} $\vee$

$\hspace{2cm}$ \texttt{\textbf{contain}(\textbf{get\_mode}(mo.Traj), Train)} $\vee$ 

$\hspace{2cm}$ \texttt{\textbf{contain}(\textbf{get\_mode}(mo.Traj), Tube)} \\
 
\end{itemize}

\begin{itemize}
 \item \textbf{Query} 8. How long does \textit{Bobby} walk during his trip? \\

$\hspace{1cm}$ \texttt{SELECT \textbf{duration}(\textbf{deftime}(\textbf{at}(mo.Traj, Walk)))}

$\hspace{1cm}$ \texttt{FROM MOHagen as mo}

$\hspace{1cm}$ \texttt{FROM mo.Name = ``Bobby"} \\
\end{itemize}


\begin{itemize}
 \item \textbf{Query} 9. Where does \textit{Bobby} walk during his trip? \\

The result can be represented in two ways: \\

$\hspace{1cm}$ \texttt{SELECT \textbf{trajectory}(\textbf{at}(mo.Traj, Walk))}

$\hspace{1cm}$ \texttt{FROM MOHagen as mo}

$\hspace{1cm}$ \texttt{FROM mo.Name = ``Bobby"} \\


or \\ 

$\hspace{1cm}$ \texttt{SELECT \textbf{ref\_obj}(\textbf{get\_infra}(\textbf{at}(mo.Traj, Walk), SpaceHagen)}, 

$\hspace{4cm}$ \texttt{SpaceHagen) }

$\hspace{1cm}$ \texttt{FROM MOHagen as mo}

$\hspace{1cm}$ \texttt{FROM mo.Name = ``Bobby"} \\
\end{itemize}

The first query should be clear and we give some comments for the second expression. First, the moving object is restricted within the mode $Walk$ by \textbf{at}. Second, the referenced objects are returned by the reference type using \textbf{get\_infra}. Third, the infrastructure objects with full representation are returned by \textbf{ref\_obj} because the reference type can't provide enough information for the places where \textit{Bobby} walks. \\

\begin{itemize}
 \item \textbf{Query} 10. Find all people staying at office room $r_{312}$ in FernUniversity for more than 2 hours? \\

\texttt{LET room\_id = ELEM(} 

$\hspace{1cm}$ \texttt{SELECT ind.In\_id}

$\hspace{1cm}$ \texttt{FROM \textbf{get\_indoor}(SpaceHagen) as ind}

$\hspace{1cm}$ \texttt{FROM ind.Name = ``FernUni-312''}) \\

Then, we create a $\underline{genloc}$ object denoted by $r\_loc(room\_id,(\bot,\bot))$ representing the office room. \\ 


$\hspace{1cm}$ \texttt{SELECT mo.Name}

$\hspace{1cm}$ \texttt{FROM MOHagen as mo}

$\hspace{1cm}$ \texttt{FROM \textbf{duration}(\textbf{deftime}(\textbf{at}(mo.Traj, r\_loc))) $>$ 120}  \\

\end{itemize}



\textbf{3. Interaction between Different Transportation Modes and Infrastructures} \\

\begin{itemize}
 \item \textbf{Query} 11. Find all buses passing the city center area. \\

Assume the name is for the area is ``CityCenter''. \\

\texttt{LET R\_Id = ELEM(}  

$\hspace{1cm}$ \texttt{SELECT rbo.Poly\_id }

$\hspace{1cm}$ \texttt{FROM \textbf{get\_rbo}(SpaceHagen) as rbo}

$\hspace{1cm}$ \texttt{FROM rbo.Name = ``CityCenter''}) \\

\texttt{LET r\_center = genloc(R\_Id, ($\bot$, $\bot$))} \\

$\hspace{1cm}$ \texttt{SELECT *}

$\hspace{1cm}$ \texttt{FROM \textbf{get\_ptn}(SpaceHagen) as ptn}

$\hspace{1cm}$ \texttt{FROM \textbf{passes}(ptn.Vehicle, r\_center, SpaceHagen)} \\

\end{itemize}

\begin{itemize}
 \item \textbf{Query} 12. Did bus 527 pass any traveller going by bicycle? \\

We define \textit{pass} by the distance between the two objects is less than 2 meters. \\

$\hspace{1cm}$ \texttt{SELECT mo.Name}

$\hspace{1cm}$ \texttt{FROM \textbf{get\_ptn}(SpaceHagen) as mbus, MOHagen as mo}

$\hspace{1cm}$ \texttt{FROM mbus.Name = ``Bus527''} $\wedge$ 

$\hspace{2cm}$ \texttt{\textbf{distance}(\textbf{trajectory}(\textbf{atperiods}(\textbf{at}(mo.Traj, Bicycle)}, 

$\hspace{8.2cm}$ \texttt{\textbf{deftime}(mbus.Vehicle)))},

$\hspace{3.8cm}$ \texttt{\textbf{trajectory}(mbus.Vehicle), SpaceHagen)} $<$ \texttt{2} 

\end{itemize}

To calculate the distance between travellers and moving buses, we project their movement into the 2D space within the same time period. \textbf{at} restricts the movement with the mode $Bicycle$ and \textbf{atperiods} gets the moving part of travellers within the same time interval as the bus 527. We use \textbf{trajectory} to get the movement curve in space and then compare the distance. \\


\begin{itemize}
 \item \textbf{Query} 13. Who entered bus 527 at bus stop ``FernUniversity''? \\

\texttt{LET bs\_stop = ELEM(} 

$\hspace{1cm}$ \texttt{SELECT bs.Stop}

$\hspace{1cm}$ \texttt{FROM \textbf{get\_bs}(SpaceHagen) as bs, \textbf{get\_br}(SpaceHagen) as br}

$\hspace{1cm}$ \texttt{FROM bs.Name = ``FernUniversity'' $\wedge$ 
br.Name = ``Bus527'' } $\wedge$ 

$\hspace{2cm}$ \texttt{bs.Bs\_id = br.Br\_id})  \\


\texttt{LET bs\_route = ELEM(} 

$\hspace{1cm}$ \texttt{SELECT br.Route}

$\hspace{1cm}$ \texttt{FROM \textbf{get\_br}(SpaceHagen) as br}

$\hspace{1cm}$ \texttt{WHERE br.Name = ``Bus527''}) \\


\texttt{LET p = geo\_data(bs\_stop, bs\_route)} \\


\texttt{LET bs\_loc = genloc($\bot$,(p.x,p.y))}  \\


We assume the travellers walk to the bus stop before entering the bus. And if the movement part before the bus stop is still $Bus$, people entered the bus before the stop ``FernUniversity'' which is not qualified. \\


$\hspace{1cm}$ \texttt{SELECT mo.Name}

$\hspace{1cm}$ \texttt{FROM MOHagen as mo}

$\hspace{1cm}$ \texttt{FROM \textbf{intersects}(bs\_loc,}

$\hspace{3cm}$ \texttt{\textbf{trajectory}(\textbf{at}(mo.Traj, Walk)), SpaceHagen)} $\wedge$ 

$\hspace{2cm}$ \texttt{\textbf{intersects}(bs\_loc,}

$\hspace{3cm}$ \texttt{\textbf{trajectory}(\textbf{at}(mo.Traj, Bus)), SpaceHagen)} $\wedge$ 

$\hspace{2cm}$ \texttt{\textbf{deftime}(\textbf{at}(mo.Traj, Walk)) $<$  \textbf{deftime}(\textbf{at}(mo.Traj, Bus))}

\end{itemize}

First, we create a $\underline{genloc}$ object denoting the bus stop. Then, we get two sub movements of travellers where one is for $Walk$ and the other is for $Bus$, which is done by \textbf{at} and \textbf{trajectory}. Finally, we apply  operator \textbf{intersects} to check whether the trajectory intersects the bus stop. And the time for the $Walk$ movement should be earlier than the time for $Bus$. This is to filter the case that people get off the bus at bus stop ``FernUniversity''. \\


\begin{itemize}
 \item \textbf{Query} 14. Did anyone who was at the University on floor X between 4:30 pm and 5 pm take a bus to the main (train) station? \\

\texttt{LET qt = period(instant(2010, 12, 5, 16, 30)}, 

$\hspace{3cm}$ \texttt{instant(2010, 12, 5, 17, 0))}.\\

Assume the name of the floor is ``FernUni-FloorX''. \\

\texttt{LET floor\_id = ELEM(} 

$\hspace{1cm}$ \texttt{SELECT ind.In\_id}

$\hspace{1cm}$ \texttt{FROM \textbf{get\_indoor}(SpaceHagen) as ind}

$\hspace{1cm}$ \texttt{FROM ind.Name = ``FernUni-FloorX''}) \\

\texttt{LET r\_{station} = ELEM(}

$\hspace{1cm}$ \texttt{SELECT AGGR(\textbf{union}, \textbf{get\_region}(ind.Room))}

$\hspace{1cm}$ \texttt{FROM \textbf{get\_indoor}(SpaceHagen) as ind}

$\hspace{1cm}$ \texttt{FROM ind.Name = ``MainStation''}) \\


$\hspace{1cm}$ \texttt{SELECT mo.Name}

$\hspace{1cm}$ \texttt{FROM MOHagen as mo}

$\hspace{1cm}$ \texttt{FROM \textbf{contain}(\textbf{get\_infra}(\textbf{atperiods}(mo.Traj, qt), SpaceHagen)}, 

$\hspace{3.5cm}$ \texttt{floor\_id)} $\wedge$ 

$\hspace{2cm}$ \texttt{\textbf{contain}(\textbf{get\_mode}(mo.Traj), Bus)} $\wedge$ 

$\hspace{2cm}$ \texttt{\textbf{intersects}(r\_station},

$\hspace{4.2cm}$ \texttt{\textbf{convert\_traj}(\textbf{trajectory}(\textbf{at}(mo.Traj, Bus))},

$\hspace{7.4cm}$ \texttt{SpaceHagen, Bus)} $\wedge$ 

$\hspace{2cm}$ \texttt{\textbf{deftime}(\textbf{at}(mo.Traj, Bus))} $<$ \texttt{\textbf{deftime}(\textbf{at}(mo.Traj, r\_station))}
\end{itemize}

The first \textbf{contain} checks whether the travellers pass the floor during the given time interval. 
The second \textbf{contain} checks whether the mode $Bus$ is included. The third predicate examines whether the \textit{bus} trajectory intersects the main (train) station where operator \textbf{convert\_traj} is used to convert the generic representation $\underline{genrange}$ to the specific type in \textit{ptn} which is $\underline{gline}$. The operator \textbf{intersects} exists in earlier system \cite{GA2006} which compares a region and a gline. The last predicate is to guarantee that the travellers first take the bus and then arrive at the station instead of taking the bus from the station. \\


\begin{itemize}
 \item \textbf{Query} 15. Who arrived by taxi at Fern University in December? \\

\texttt{LET qt = period(Day(2010, 12, 1), Day(2010, 12, 31))}. \\

First, we restrict moving objects with transportation modes $Taxi$ and $Indoor$ in December. \\

\texttt{LET MOHagen1 =}

$\hspace{1cm}$ \texttt{SELECT *}

$\hspace{1cm}$ \texttt{FROM MOHagen as mo}

$\hspace{1cm}$ \texttt{WHERE \textbf{contain}(\textbf{get\_mode}(\textbf{atperiods}(mo.Traj, qt)), Taxi)} $\wedge$ 

$\hspace{2.2cm}$ \texttt{\textbf{contain}(\textbf{get\_mode}(\textbf{atperiods}(mo.Traj, qt)), Indoor)} \\
 

Second, we restrict the $Indoor$ part only in Fern University, but not in some other buildings. 
Assume the name for the university is $s\_name$ = ``FernUniversity in Hagen''. \\

\texttt{LET Uni = } 

$\hspace{1cm}$ \texttt{SELECT *}

$\hspace{1cm}$ \texttt{FROM \textbf{get\_indoor} ( SpaceHagen) as ind}

$\hspace{1cm}$ \texttt{WHERE ind.Name = s\_name} \\

We collect moving objects that their movement cover the university. \\

\texttt{LET MOHagen2 =} 

$\hspace{1cm}$ \texttt{SELECT *}

$\hspace{1cm}$ \texttt{FROM MOHagen1 as mo, Uni as uni}

$\hspace{1cm}$ \texttt{WHERE \textbf{contain}(\textbf{get\_infra}(\textbf{at}(\textbf{atperiods}(mo.Traj, qt), Indoor)}, 

$\hspace{4cm}$ \texttt{uni.In\_id)}\\

\end{itemize}


Third, we find the qualified moving objects by \\


$\hspace{1cm}$ \texttt{SELECT mo.Name }

$\hspace{1cm}$ \texttt{FROM MOHagen2 as mo}

$\hspace{1cm}$ \texttt{WHERE \textbf{intersects}(}


$\hspace{3.8cm}$ \texttt{\textbf{trajectory}(\textbf{at}(\textbf{atperiods}(mo.Traj, qt), Taxi))},  

$\hspace{3.8cm}$ \texttt{\textbf{trajectory}(\textbf{at}(\textbf{atperiods}(mo.Traj, qt), Indoor)),}

$\hspace{3.8cm}$ \texttt{SpaceHagen)} $\wedge$ 

$\hspace{2.1cm}$ \texttt{\textbf{deftime}(\textbf{at}(\textbf{atperiods}(mo.Traj, qt), Taxi))} $<$ 

$\hspace{2.1cm}$ \texttt{\textbf{deftime}(\textbf{at}(\textbf{atperiods}(mo.Traj, qt), Indoor))}  \\


There are two predicates (\textbf{intersects} and $<$) in the WHERE clause. The first checks whether the two sub trajectories ($Taxi$ and $Indoor$) intersect and the second is to guarantee that the $Taxi$ movement should be earlier than the $Indoor$ movment. This is to filter the case that people leave the university and then take a taxi. 
