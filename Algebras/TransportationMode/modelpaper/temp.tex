\newpage 

\begin{itemize}
 \item The symbol data type is introduced in Definition \ref{infraobjectsymbol} of Section \ref{sec:infrastructure}. 
 \item The data type representing trajectories ($\underline{genrange}$) is in Definition \ref{genrange} of Section \ref{sec:location}. 
 \item The definition for two units which are \textit{mergeable} is in Section \ref{sec:trajectoryrepresentation}.  
 \item I write a paragraph explaining bus trajectory and traveller's (taking the bus) trajectory in Section \ref{sec:ioinptn}. 
 \item Operators on transportation modes are in Table \ref{tab:operators4} in Section \ref{sec:infrastructureandtm}. Operators on coverting generic data types to specific types are in Table \ref{tab:convertoperators} in Section \ref{sec:datatypeconvert}. Operators on specifc infrastructures are in Table \ref{tab:operators3} in Section \ref{sec:specific}. 

  \item A short paragraph about implementation is added in Section \ref{sec:implementation}.
 \item The notations used for queries are introduced in the beginning of Section \ref{sec:exampledataandquery}.
\end{itemize}


\textbf{Issues from Ralf} 

\begin{itemize}
 \item Can we relate locations within different infrastructures to each other? For example, to find objects traveling by bus through a region in real space? \\
 
 Yes, we can. In this case, we need the \textit{space} object to transform locations in the same coordinate/reference system, i.e., in the free space.  

 \item What about aggregations of infrastructure objects such as the set of rooms on a floor or the set of pavement regions along a road? \\

  After doing union a set of rooms on a floor, we get a groom object where its elements (3D regions) are individual rooms. In this case, only one object identifier will be used. So, the movement of a person walking through several rooms on a floor is represented by referencing to one groom. It is the same for doing union on a set of pavement regions. 

 \item Switching points between infrastructures. How can we refer to them in a query? \\

  First, we can get the 2D point. And then, we can create a genloc object for that point. See the Query 5 and 13.\\

 \item Consider infrastructures as a space where each infrastructure is one location. Similar consider transportation modes as a space. \\

 Is this a question? \\ 

 \item Rough representations of trajectoreis. \\

 An operator called \textbf{lowres} is introduced in Table \ref{tab:operators2} in Section \ref{sec:temporaltypes}, which is to get the moving object with coarse location where the units are low resolution temporal units. And then we can apply operator \textbf{trajectory} to get the rough trajectory. 

 \item Access to genloc or genrange values. \\

      An operator called \textbf{ref\_id} is introduced in Table \ref{tab:operators1} in Section \ref{sec:non-temporalop}.
\end{itemize}
