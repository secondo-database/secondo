\section{A Relation Interface}
\label{sec:relationinterface}
\subsection{Space: A Relational View}
\label{sec:spacerel}
We provide an interface to exchange information between values of the proposed data types and a relational environment. To manage moving objects,  one first has to supply  
\textit{infrastructure objects} that moving objects \textit{reference} to. For each kind of 
infrastructure, we create a relation where each tuple corresponds to an infrastructure object. The schemas for all infrastructures are described in Table \ref{tab:infrarel}.
Relation $rel_{busstop}$, $rel_{busroute}$ store bus stops and bus routes, respectively, and  $rel_{ptn}$ stores public transportation vehicles (here we only consider 
public buses). $rel_{indoor}$ is for the \textit{indoor} environment, 
$rel_{rbo}$ collects the partitioned zones of an \textit{outdoor} space, and 
$rel_{rn}$ records the routes for a road network. 
As $I_{fs}$ is an empty set, there is no relation for it. 
For each relation, the data type representing \textit{infrastructure objects} is embedded as 
an attribute. To have the unique identifier for each object, for each kind of infrastructure we define a range of int values for the object id and there is no overlap between the ranges for different infrastructures. \\ 

\begin{table}[ht]
 \begin{center} 
  \begin{tabular}{c|c|c}
	\hline
	 Name&Infrastructure&Relation Schema \\
	\hline
    $rel_{busstop}$ & $ptn$ & $(Bs\_id:int$, $Stop:busstop$, $Name$ \\
    \hline
    $rel_{busroute}$ & $ptn$ &$(Br\_id:int$, $Route:busroute$, $Name$, $Up:bool$) \\
    \hline 
	$rel_{ptn}$ & $ptn$ &$(Mp\_id:int$, $Vehicle:mpptn$, $Name$ \\
	\hline
	$rel_{indoor}$& $indoor$ &$(In\_id:int$, $Room:groom$, $Name$ \\
	\hline
	$rel_{rbo}$& $rbo$ &$(Poly\_id:int$, $Reg:region$, $Name$ \\
	\hline 
	$rel_{rn}$& $rn$ &$(R\_id:int$, $Road:line$, $Name$\\
	\hline 
  \end{tabular}
 \end{center}
 \caption{\label{tab:infrarel} Infrastructure Relations}
\end{table}


With the relations for all infrastructures, we can construct the space by applying the operation: \\

$\underline{rel} \times \underline{rel} \times \underline{rel} \times \underline{rel} \times \underline{rel} \times \underline{rel} \rightarrow \underline{space}$ \hspace{1cm} \textbf{createspace}\\

The operator takes in the relevant information from each infrastructure described by a relation.
With the relational interface, we can exchange information between values of the proposed data types and a relational environment. We can get each infrastructure relation information by: \\

$\underline{space} \rightarrow \underline{rel}$ \hspace{1cm} \textbf{get\_bs}, \textbf{get\_br}, \textbf{get\_ptn}, \textbf{get\_indoor}, \textbf{get\_rbo}, \textbf{get\_rn} \\

For each infrastructure, one can retrieve the required infrastructure objects by operations of a relational environment. In the following, we use an example to show how the interface works. Suppose we have the relations for the \textit{infrastructure} data of a city called \textit{Hagen}: \\


(1) $rel_{busstop}^{Hagen}$, $rel_{busroute}^{Hagen}$: bus stops and bus routes in \textit{Hagen}; \\

(2) $rel_{ptn}^{Hagen}$: public buses in $Hagen$; \\

(3) $rel_{indoor}^{Hagen}$: rooms of buildings in $Hagen$;  \\

(4) $rel_{rbo}^{Hagen}$: a set of partitioned regions covering $Hagen$; \\

(5) $rel_{rn}^{Hagen}$: roads and streets in $Hagen$. \\

Then, we can create \\

\texttt{SpaceHagen} = \textbf{createspace} ($rel_{busstop}^{Hagen}$, $rel_{busroute}^{Hagen}$, $rel_{ptn}^{Hagen}$, $rel_{indoor}^{Hagen}$, $rel_{rbo}^{Hagen}$, $rel_{rn}^{Hagen}$) \\

representing the infrastructure $space$ for \textit{Hagen}. Some example queries are listed below.\\

\textbf{Examples}\\
\begin{itemize}
 \item Show me the information of \textit{Alexander} street.  \\

$\hspace{1cm}$ \texttt{SELECT *}

$\hspace{1cm}$ \texttt{FROM \textbf{get\_rn}(SpaceHagen) as rn}

$\hspace{1cm}$ \texttt{WHERE rn.Name = ``Alexander"} \\

\item Where does ``Bus512`` go?  \\

$\hspace{1cm}$ \texttt{SELECT *}

$\hspace{1cm}$ \texttt{FROM \textbf{get\_ptn}(SpaceHagen) as mbus}

$\hspace{1cm}$ \texttt{WHERE mbus.Name = ``Bus512"} \\

\item Find all areas intersect \textit{Alexander} street.  \\

$\hspace{1cm}$ \texttt{SELECT *} 

$\hspace{1cm}$ \texttt{FROM \textbf{get\_rn}(SpaceHagen) as rn, \textbf{get\_rbo}(SpaceHagen) as rbo}

$\hspace{1cm}$ \texttt{WHERE rn.Name = ``Alexander" $\wedge$ \textbf{intersects}(rn.Road, rbo.Reg)} \\

\end{itemize}

Assume we also have some trajectory data of citizens living and working in \textit{Hagen}. We use the object-based approach \cite{DBG2009} to store moving objects where the complete history is kept together. The trip is 
stored as an attribute in a relation named \texttt{MOHagen} with the schema: \\

$(Mo\_id:int$, $Traj:genmo$, $Name$ \\

It has three attributes: $Mo\_id$ being the moving object identifier, $Traj$ storing the 
trajectory data and a string value describing the name. In the next section, we show 
more queries on the model based on the above example relations. 

\subsection{Graph Model for Indoor Navigation}
\label{sec:relgraphmodel}
We use two relations to store graph nodes and edges, named $rel_{indoor}^n$ and $rel_{indoor}^e$, respectively. According to Definition \ref{sec:navigation},
the node relation stores doors and the edge relation stores grooms. 
The edge relation is derived from $rel_{indoor}$ with extensions on weight value and the path. 
Table \ref{tab:indoorgraphrel} shows their schemas. \\

\begin{table}[ht]
 \begin{center} 
  \begin{tabular}{c|c|c}
	\hline
	 Name&Infrastructure&Relation Schema \\
	\hline
	$rel_{indoor}^n$& $Indoor$ & $(D\_id:int$, $Connect:door)$ \\
	\hline
	$rel_{indoor}^e$& $Indoor$ & $(In\_id:int$, $Room:groom$, $Name$ \\
	& &$Weight:real$, $Path:line3d)$ \\
	\hline
  \end{tabular}
 \end{center}
 \caption{\label{tab:indoorgraphrel} Indoor Graph Relations}
\end{table}

We can create the graph \textbf{GMI} (defined in Section \ref{sec:navigation}) by the following operator. \\

$\underline{rel} \times \underline{rel} \rightarrow GMI$ \hspace{0.3cm} \textbf{createindoorgraph} \\

Then, we can run shortest path queries by \\

$\underline{genloc} \times \underline{genloc} \times \underline{int} \times \underline{instant} \times GMI\rightarrow \underline{line3d}$ \hspace{0.3cm} \textbf{indoornavigation} \\

$\underline{genloc} \times \underline{genloc} \times \underline{int} \times \underline{instant} \times GMI\rightarrow set(\underline{groom})$ \hspace{0.3cm} \textbf{indoornavigation} \\

where the first two arguments are the start and end location, $\underline{int}$ denotes the type of the shortest path (e.g., distance, time), $\underline{instant}$ be the query issue time and \textbf{GMI} is the graph. We can use data type $\underline{line3d}$ representing the result which is a 3D curve in space. It is also possible to describe the path in a rough way that is by the grooms passed. 

\subsection{Implementation}
\label{sec:implementation}
The earlier work in \cite{FG+00,GA2006} has been implemented in an extensible database system SECONDO \cite{GAABDHHS05} and the infrastructures there are free space and road network. The generic model represents moving objects in multiple environments which covers the two infrastructures and the representation is consistent with 
before. So, the generic model can still be implemented in the database system. We omit discussing the implementation technique in detail. 
