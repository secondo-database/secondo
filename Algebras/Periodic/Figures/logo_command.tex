% This code snipped provides the LaTeX representation of 
% logo of the FernUniversitaet in Hagen. 
% They are 5 parameters:
%  #1 = width of the logo, e.g. 10cm
%  #2 #3 #4 The basic color in rgb model . all values must be in range 0-1
%  #5 number of repeatations for creating the shaded pen
%     The minimum value has to be 1 
%
%  Before you can use this macro, you have to insert some packages, namely
%  fp, pstricks, multido
%  Note that the fp package must be the first one in the import list
\newcommand{\Unilogo}[5]
{\newdimen\LogoWidth
\LogoWidth=#1
\newrgbcolor{BC}{{#2} {#3} {#4}} % define the basic color
\FPset\circnumber{#5} % this definition determines the granularity of the gradient
\FPsub{1}{#2}{\DiffR}
\FPdiv{\DiffR}{\DiffR}{\circnumber}
\FPsub{1}{#3}{\DiffG}
\FPdiv{\DiffG}{\DiffG}{\circnumber}
\FPsub{1}{#4}{\DiffB}
\FPdiv{\DiffB}{\DiffB}{\circnumber}
\FPdiv{\DiffRad}{70}{\circnumber}
\scaleboxto(\LogoWidth,0){
\psset{unit=1pt}
\begin{pspicture}(0,0)(420,420)
\rput[bl]{0}(210,210){
\psarc[linewidth=20,linecolor=BC](0,0){100}{180}{0}
\psarc[linewidth=20,linecolor=BC](0,0){150}{180}{0}
\psarc[linewidth=20,linecolor=BC](0,0){200}{180}{0}
\pspolygon[linestyle=none,fillstyle=solid,fillcolor=BC](-110,0)(-110,170)(-90,200)(-90,0)
\pspolygon[linestyle=none,fillstyle=solid,fillcolor=BC](110,0)(110,170)(90,200)(90,0)
\pspolygon[linestyle=none,fillstyle=solid,fillcolor=BC](-160,0)(-160,120)(-140,150)(-140,0)
\pspolygon[linestyle=none,fillstyle=solid,fillcolor=BC](160,0)(160,120)(140,150)(140,0)
\pspolygon[linestyle=none,fillstyle=solid,fillcolor=BC](-210,0)(-210,10)(-190,100)(-190,0)
\pspolygon[linestyle=none,fillstyle=solid,fillcolor=BC](210,0)(210,10)(190,100)(190,0)
% paint a shaded  circle
\begin{psclip}{\pscircle[linestyle=none,fillstyle=solid,fillcolor=BC](0,0){70}}
	\multido{\rR=#2+\DiffR,\rG=#3+\DiffG,\rB=#4+\DiffB,\rr=70+-\DiffRad}{\circnumber}{ 
			\newrgbcolor{c}{{\rR} {\rG} {\rB}}
			\pscircle[linestyle=none,fillstyle=solid,fillcolor=c](17,17){\rr}
	}
\end{psclip}
}
\end{pspicture}
}
}
